\documentclass{resume}
\usepackage{palatino}
\usepackage{hyperref}
\hypersetup{
    colorlinks=true,
    linkcolor=blue,
    filecolor=magenta,      
    urlcolor=cyan,
}
\usepackage[left=0.4in,top=0.3in,right=0.4in,bottom=0.3in]{geometry} % Document margins
\newcommand{\tab}[1]{\hspace{.2667\textwidth}\rlap{#1}}
\newcommand{\itab}[1]{\hspace{0em}\rlap{#1}}
\name{Jerome Wei} % Your name
%\address{508 Ramona Ave. Albany, California, 94706} % Your address
%\address{123 Pleasant Lane \\ City, State 12345} % Your secondary addess (optional)
\address{(510)-833-0536 $|$ jeromew@berkeley.edu $|$ \href{https://github.com/jeromew21}{github} $|$ \href{https://www.linkedin.com/in/jeromewei/}{linkedin}}
 % Your phone number and email

\begin{document}

%----------------------------------------------------------------------------------------
%	EDUCATION SECTION
%----------------------------------------------------------------------------------------

\begin{rSection}{Education}

{\bf University of California, Berkeley} \hfill {\em 2017-2022} 
\\ B.A. Computer Science
%Minor in Linguistics \smallskip \\
%Member of Eta Kappa Nu \\
%Member of Upsilon Pi Epsilon \\
\end{rSection}
%----------------------------------------------------------------------------------------
%	TECHNICAL STRENGTHS SECTION
%----------------------------------------------------------------------------------------

\begin{rSection}{Skills}

\begin{tabular}{ @{} >{\bfseries}l @{\hspace{6ex}} l }
Programming Languages &  C/C++, Python, Java, Javascript \\
Software \& Tools & Unix, Git, PyTorch/TensorFlow, AWS, CMake \\
\end{tabular}

\end{rSection}

%----------------------------------------------------------------------------------------
%	WORK EXPERIENCE SECTION
%----------------------------------------------------------------------------------------

\begin{rSection}{Relevant Experience}

\begin{rSubsection}{Amazon Web Services}{May 2023 - November 2023}{SDE I}{}
\item Joined IoT RoboRunner team.
\item Wrote unit tests, diagnosed and addressed issues in CI/CD pipeline, and conducted code reviews.
\end{rSubsection}

\begin{rSubsection}{Amazon Web Services}{May 2022 - August 2022}{SDE Intern}{}
\item Implemented and delivered both milestones of intern project with AWS Robotics team involving AWS Greengrass, internal CI/CD, and robotics middleware.
\item Delivered detailed design document including precise customer requirements, implementation plan, and high-level architecture diagrams.
\end{rSubsection}

\begin{rSubsection}{University of California, San Francisco}{August 2020 - January 2021}{Intern, Keiser Lab}{}
\item Responsible for writing scalable and portable data preprocessing scripts, and performing data analysis.
\item Trained model on histopathological data and ran experiments to understand model robustness, effects of artifacts and blur, and interpretability.
\end{rSubsection}

\begin{rSubsection}{Lawrence Berkeley National Laboratory}{January 2019 - November 2019}{Undergraduate Student Assistant}{}
\item Researched novel ways to speed up building energy use simulation software. 
\item Wrote framework 
to test refactored methods, track accumulated
error, and log memoization properties such as miss rate and hash collision rate. 
\item Achieved up to 300\% speedup on select functions. 
\end{rSubsection}


%------------------------------------------------

\begin{rSubsection}{Computer Science Mentors}{January 2019 - May 2019}{Junior Mentor, CS70 (Discrete Mathematics and Probability Theory)}{}
\item Led weekly mentoring groups for CS70. 
\item Sections focus on solidifying understanding of concepts covered in lecture and discussion. 
\item Prepared weekly lesson plans that provide coverage of material and cater towards individual learning
styles.
\end{rSubsection}

% \begin{rSubsection}{Map2Next}{August 2018 - December 2018}{Intern, Software Engineering}{}
% \item Designed framework for automatically scraping information from web pages into abridged format.
% \item Scaled with parallel downloads and used Selenium for cases where HTML parsing wasn't sufficient.
% \item 
% \end{rSubsection}

\end{rSection}



%	EXAMPLE SECTION
%----------------------------------------------------------------------------------------

\begin{rSection}{Selected Projects} \itemsep -2pt
\item \textbf{Crossword Solver} Crossword puzzle desktop GUI and backend written in C++ able to solve small crossword puzzles using custom backtracking algorithm.
\item \textbf{Chess Engine} A fully functional chess engine, written in C++ from scratch. Strength of 2000 ELO based on average performance against other engines.
% \item \textbf{"Wikigame" Solver} A method to scrape Wikipedia with breadth-first search and the resulting graph to find shortest path between articles.
% \item \textbf{ChocoPy Compiler} A compiler for a statically typed dialect of Python 3, for CS164. Worked in a small group on three stages: lexer, parser, and code generator.


\end{rSection}

%----------------------------------------------------------------------------------------
% \begin{rSection}{Relevant Courses}
%  \itab{Data Structures} \tab{}  \tab{Linear Algebra}
% \\ \itab{Discrete Math} \tab{}  \tab{Multivariable Calculus} 
% \\ \itab{Probability Theory} \tab{}  \tab{Programming Languages and Compilers}
% \\ \itab{Efficient Algorithms and Intractable Problems} \tab{} \tab{Machine Structures} 
% \\ \itab{Process Control (ongoing)} \tab{} \tab{Electrodynamics}
% \end{rSection}


\end{document}